\documentclass[12pt]{article}
\usepackage{amsmath}
\usepackage{graphicx}
\usepackage{float}
\usepackage[backend=biber]{biblatex}
\usepackage{caption}
\usepackage[colorlinks=true,linkcolor=blue,urlcolor=black,bookmarksopen=true]{hyperref}
\usepackage{hyperref}
\usepackage{cleveref}
\usepackage{multirow}

%\addbibresource{~/.config/LatexStuff/bibtex/bib/physics.bib}
\addbibresource{sample.bib}
\graphicspath{{./images/}}
\setlength{\parindent}{0pt}

\title{Frank Hertz Experiment}
\author{Robert James Stevenson}
\date{24 August 2023}

\newcommand{\mycomment}[1]{}

\begin{document}
\mycomment{
	label for EQS \label{}
	citing reference \cite{}
	equation

	\begin{equation}
		\label{eqn:somelabel}
		e=mc^2
	\end{equation}
}


\maketitle

\section{Abstract}
In this experiment, we investigate the Franck-Hertz setup and its essential findings.
Through the careful control of accelerating voltages and using the resulting current drops,
we aim to confirm the existence of discrete energy levels within mercury atoms. Our analysis yields a close agreement between experimental and theoretical values,
demonstrating the consistency of our results with established scientific principles.
The Franck-Hertz experiment continues to stand as a cornerstone in our understanding of quantum phenomena and atomic structure.

\section{Introduction}
The Franck-Hertz experiment, named after physicists James Franck and Gustav Hertz,
completely changed our understanding of atomic and subatomic behavior in the early 20th century.
By bombarding gaseous atoms with electrons and observing their response,
it provided crucial evidence for the quantization of energy levels within atoms.
This landmark experiment affirmed the wave-particle duality of electrons and played an important in shaping the foundations of quantum mechanics.
In this experiment, we'll delve into the experiment's setup and key findings. \cite{brief}

\section{Theory}

\begin{figure}[H]
	\includegraphics[width=10cm]{sketch}
	\centering
	\caption{Schematic of Experiment}
	\label{fig:sketch}
\end{figure}

\par
In this Experiment a vacuum tube, containing Mercury is heated to a temperature
such it is turned to vapor. The vacuum tube also contains a filament/cathode (labeled as K),
a positively
charged mesh grid (G1), negatively charged grid (G2) and an Anode (labeled as A)
acting  as  a collecting plate, see
\textbf{\autoref{fig:sketch}}
for circuit diagram.
Electrons are emitted from the filament and accelerated, due to the positively
charged grid (G1), through the mercury vapor, some will overcome the repulsive force due to G2,
collect on the collecting plate, which is confirmed by the ammeter.
As a result the accelerated electrons will preform elastic collisions with the
electrons belonging to the Mercury atoms.
\\
The Ampere reading is proportional to the potential across the controlled grid (G1).
\\
\\
The Kinetic Energy (KE) of the electrons is directly proportional to the potential
difference across G1, potential across G1 will hence be referred to as VG1.
When the electrons have a higher enough KE, high enough VG1, they will
then preform inelastic collisions with Mercury's electrons. The emitted electrons that
collided with the mercury atoms
will have a lower kinetic energy and less of them will be able to overcome
the repulsive force of G2, this will be indicated by a sharp drop in the ampere reading.
\\
Inelastic collisions occur as the emitted electrons now have enough energy to
excite the Mercury electrons to the next energy level. The potential of VG1 at which this phenomenon occurs will
be referred to as VBr, which is 4.9 Volts.\cite{lecture}  \cite{fth}
\\
\\
This drop in ampere reading will now occur at every scalar multiple of VBr:
$ V_{Br1} = 2V_{Br} - V_{Br} = \Delta V \\
V_{Br2} = 3V_{Br} - 2V_{Br} = \Delta V \\
V_{Br3} = 5V_{Br} - 4V_{Br} = \Delta V \\
\vdots \\
V_{BR} = (n+1)V_{BR} - (n)V_{Br} = \Delta V \\
$
\\
This shows that mercury has distinct energy levels and will absorb energy in quantized
amounts:
\begin{equation}
   E = e\Delta V = hf = \frac{hc}{\lambda}
    \label{eq:temp}
\end{equation}
$E=$ Energy (Joules [J] or Electron Volts [eV])\\
$e=$ charge of electron (Coulombs [C]) \\
$h=$ Planck's constant \\
$f=$ Frequency of emitted radiation (hertz [hz]) \\
$\lambda=$ Wavelength of emitted radiation(meters [m]) \\
$c=$ Speed of light in a vacuum (meters per second [m/s])\\
$\Delta V$ = Difference in potential (Volts [V])\\

Note:
\begin{equation}
    1eV \approx 1.602176208 \cdot 10^{-19}J
\end{equation}

\section{Procedure}



The Follwing equipment was used for the experiment:
\subsection{Apparatus}
\begin{itemize}
    \item Oven (which contains the mercury tube)
    \item thermal-control unit
    \item Control Unit (controls the oven heating element, accelerating voltage, 
        decelerating voltage and cathode)
    \item Oscilloscope
    \item connection leads
\end{itemize}

\subsection{Method}
\begin{enumerate}
    \item Make sure all the dials on the control unit are at their lowest setting.

    \item Connect the thermal control unit to the oven by inserting the thermometer into the oven

    \item Connect the oven to the control unit, the oscilloscope to the control unit, and set the
        relevant settings on the control unit as seen in
        \textbf{\autoref{fig:oven}}, \textbf{\autoref{fig:control}} and \textbf{\autoref{fig:osc}}

    \item Set the thermal control unit to 175$^{\circ}$C, then turn on the oven,
        however make sure the temperature does not exceed 200$^{\circ}$C.

    \item Once the temperature is stable at 175$^{\circ}$C, Turn on the control unit and oscilloscope.

    \item On the control unit set the heater dial to its maximum, wait 90 seconds.
        Set the Amplitude and Reverse bias dials to the indicated value.

    \item Set the Manual/Ramp switch to ``Ramp'' and turn the acceleration dial slowly, however don't
        set further than the value indicated on the control unit. An Image should appear on the oscilloscope.

    \item Measure the voltage and corresponding ampere for each ``peak'' displayed by the oscilloscope,
        see \textbf{\autoref{fig:ep}} for the expected result. 
        Then take the difference in voltage of each measurement with the successive measurement.
        This will be your VBr.
\end{enumerate}



\begin{figure}[H]

	\includegraphics[width=5cm]{oven}
	\centering
	\caption{Oven Used in experiment}
	\label{fig:oven}
	\includegraphics[width=7cm]{ControlUnit}
	\centering
	\caption{Control Unit}
	\label{fig:control}
	\includegraphics[width=7cm]{Osc}
	\centering
	\caption{Oscilloscope}
	\label{fig:osc}
\end{figure}
\\
\begin{figure}
    \includegraphics[width=5cm]{ep}
    \centering
    \caption{Theoretical Voltage Ampere Graph}
    \label{fig:ep}
\end{figure}



\mycomment{
\begin{figure}
    \centering
    \subfigure(a){\includegraphics[width=3cm]{oven}}
	\subfigure(b){\includegraphics[width=3cm]{ControlUnit}}
    \subfigure(c){\includegraphics[width=3cm]{Osc}}
    \caption{(a) blah (b) blah (c) blah (d) blah}
    \label{fig:foobar}
\end{figure}
}

\section{Results and Discussion}
Referring to \textbf{\autoref{fig:osc2}}, The x-axis is Voltage (0.5 volts per division)
and the y-axis is Current, the unit of current is not important as the dips current determines which Voltage value to measure.
Ignoring the bottom graph (which is noise), the top graph's shape and trend is similar to theoretical graph,
\textbf{\autoref{fig:ep}}
\\
\\
Referring \textbf{\autoref{tab:table}}, which is the result of tabulating the experimental and theoretical voltage ampere graphs,
The Experimental values did not stay constant throughout each peak to peak difference and the average of these values did not match or come close (more than 10 \%) to the theoretical value.
This inconsistency across the data is attributed to the oven having a binary heating element (controlled by a relay instead of a thyristor to vary the sine wave of the current)
, fully on or fully off (no in between),
and the oven having a high temperature threshold activation, the oven had no way of varying the current
flowing through the heating element to keep the temperature constant.
\\
Instead, the oven temperature would drop below the specified temperature until it reached
the threshold temperature which would activate the relay and let current pass through the heating element until the upper threshold was reached. This would
have an effect on the mercury vapor.
\\
\\
\\ The Error of the experimental value can be calculated:
\begin{equation}
    \label{eq:error}
    \begin{split}
        \%error = & \frac{|theoretical - \overline{experimental}|}{theoretical} * 100\% \\
        \%error = & \frac{|4.9ev - 4.53ev|}{4.9ev} * 100\% = 7.55\%
    \end{split}
\end{equation}
Factors that attributed to this low error of 7.55\% are oxidation on the terminals inside the equipment (experiment took place near the coast), un-calibrated or wrongly calibrated equipment,
and human error.
\\
\\
The wavelength of excited electrons can be calculated:
\begin{equation}
    \label{eq:wave}
    \begin{split}
        \lambda = & \frac{hc}{E[joules]} = \frac{hc}{e \overline{\Delta V}} \\
        = & \frac{hc}{e\cdot 4.533} \approx 2.735 \cdot 10^{-7}m = 273.51 nm
    \end{split}
\end{equation}
\\
Thus, the experimental wavelength was calculated to be 273.51 nm $\pm$ 7.55\%
and the theoretical value
253.7 nm falls into this error bound.
\\
\\
If the excitation energy and wavelength were given \textbf{\autoref{eq:wave}} could be restructed to 
calculate planks constant.

\begin{figure}[H]
    \begin{center}
        \includegraphics[width=8cm\textwidth]{osc2}
    \end{center}
    \caption{Experimental Voltage Ampere Graph obtained from oscilloscope}
    \label{fig:osc2}
\end{figure}

\begin{table}[H]
    \centering
    \begin{tabular}{|l|l|l|}
    \hline
    \multirow{2}{*}{Peak} & \multicolumn{2}{c|}{Difference of Accelerating Voltage between each Successive Peak [ev]} \\\cline{2-3} 
        & Experimental & Theoretical \\ \hline
        1 to 2 & 4 & 4.9 \\ \hline
        2 to 3 & 4.5 & 4.9 \\ \hline
        3 to 4 & 5.1 & 4.9 \\ \hline
        Average & 4.533333333 & 4.9 \\ \hline
        Standard Deviation & 0.550757055 & 0 \\ \hline
    \end{tabular}
    \caption{The tabulation of Theoretical (\autoref{fig:ep}) and Experimental (\autoref{fig:osc2}) values of the peak to peak voltage difference}
    \label{tab:table}

\end{table}

\section{Conclusion}
By accelerating electrons through mercury vapor and measuring the resulting voltage differences, 
this experiment sought to confirm the existence of discrete energy levels within the mercury atoms and calculate the corresponding wavelengths of emitted radiation.
\\
The measured average voltage difference between successive peaks, corresponding to the excitation energy, was found to be approximately 4.53 eV.
This value was notably close to the theoretical value of 4.9 eV, with an error of approximately 7.55\%
\\
Furthermore, the calculated wavelength of the emitted radiation was approximately 273.51 nm, with an error within 
the same range as the excitation energy. Importantly, the theoretical wavelength of 253.7 nm fell within this error margin, 
reinforcing the consistency of the results with established scientific principles.
\\

\section{References}
\printbibliography


\end{document}
